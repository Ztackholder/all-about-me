% Options for packages loaded elsewhere
% Options for packages loaded elsewhere
\PassOptionsToPackage{unicode}{hyperref}
\PassOptionsToPackage{hyphens}{url}
\PassOptionsToPackage{dvipsnames,svgnames,x11names}{xcolor}
%
\documentclass[
  letterpaper,
  DIV=11,
  numbers=noendperiod]{scrreprt}
\usepackage{xcolor}
\usepackage{amsmath,amssymb}
\setcounter{secnumdepth}{5}
\usepackage{iftex}
\ifPDFTeX
  \usepackage[T1]{fontenc}
  \usepackage[utf8]{inputenc}
  \usepackage{textcomp} % provide euro and other symbols
\else % if luatex or xetex
  \usepackage{unicode-math} % this also loads fontspec
  \defaultfontfeatures{Scale=MatchLowercase}
  \defaultfontfeatures[\rmfamily]{Ligatures=TeX,Scale=1}
\fi
\usepackage{lmodern}
\ifPDFTeX\else
  % xetex/luatex font selection
\fi
% Use upquote if available, for straight quotes in verbatim environments
\IfFileExists{upquote.sty}{\usepackage{upquote}}{}
\IfFileExists{microtype.sty}{% use microtype if available
  \usepackage[]{microtype}
  \UseMicrotypeSet[protrusion]{basicmath} % disable protrusion for tt fonts
}{}
\makeatletter
\@ifundefined{KOMAClassName}{% if non-KOMA class
  \IfFileExists{parskip.sty}{%
    \usepackage{parskip}
  }{% else
    \setlength{\parindent}{0pt}
    \setlength{\parskip}{6pt plus 2pt minus 1pt}}
}{% if KOMA class
  \KOMAoptions{parskip=half}}
\makeatother
% Make \paragraph and \subparagraph free-standing
\makeatletter
\ifx\paragraph\undefined\else
  \let\oldparagraph\paragraph
  \renewcommand{\paragraph}{
    \@ifstar
      \xxxParagraphStar
      \xxxParagraphNoStar
  }
  \newcommand{\xxxParagraphStar}[1]{\oldparagraph*{#1}\mbox{}}
  \newcommand{\xxxParagraphNoStar}[1]{\oldparagraph{#1}\mbox{}}
\fi
\ifx\subparagraph\undefined\else
  \let\oldsubparagraph\subparagraph
  \renewcommand{\subparagraph}{
    \@ifstar
      \xxxSubParagraphStar
      \xxxSubParagraphNoStar
  }
  \newcommand{\xxxSubParagraphStar}[1]{\oldsubparagraph*{#1}\mbox{}}
  \newcommand{\xxxSubParagraphNoStar}[1]{\oldsubparagraph{#1}\mbox{}}
\fi
\makeatother


\usepackage{longtable,booktabs,array}
\usepackage{calc} % for calculating minipage widths
% Correct order of tables after \paragraph or \subparagraph
\usepackage{etoolbox}
\makeatletter
\patchcmd\longtable{\par}{\if@noskipsec\mbox{}\fi\par}{}{}
\makeatother
% Allow footnotes in longtable head/foot
\IfFileExists{footnotehyper.sty}{\usepackage{footnotehyper}}{\usepackage{footnote}}
\makesavenoteenv{longtable}
\usepackage{graphicx}
\makeatletter
\newsavebox\pandoc@box
\newcommand*\pandocbounded[1]{% scales image to fit in text height/width
  \sbox\pandoc@box{#1}%
  \Gscale@div\@tempa{\textheight}{\dimexpr\ht\pandoc@box+\dp\pandoc@box\relax}%
  \Gscale@div\@tempb{\linewidth}{\wd\pandoc@box}%
  \ifdim\@tempb\p@<\@tempa\p@\let\@tempa\@tempb\fi% select the smaller of both
  \ifdim\@tempa\p@<\p@\scalebox{\@tempa}{\usebox\pandoc@box}%
  \else\usebox{\pandoc@box}%
  \fi%
}
% Set default figure placement to htbp
\def\fps@figure{htbp}
\makeatother





\setlength{\emergencystretch}{3em} % prevent overfull lines

\providecommand{\tightlist}{%
  \setlength{\itemsep}{0pt}\setlength{\parskip}{0pt}}



 


\KOMAoption{captions}{tableheading}
\makeatletter
\@ifpackageloaded{bookmark}{}{\usepackage{bookmark}}
\makeatother
\makeatletter
\@ifpackageloaded{caption}{}{\usepackage{caption}}
\AtBeginDocument{%
\ifdefined\contentsname
  \renewcommand*\contentsname{Table of contents}
\else
  \newcommand\contentsname{Table of contents}
\fi
\ifdefined\listfigurename
  \renewcommand*\listfigurename{List of Figures}
\else
  \newcommand\listfigurename{List of Figures}
\fi
\ifdefined\listtablename
  \renewcommand*\listtablename{List of Tables}
\else
  \newcommand\listtablename{List of Tables}
\fi
\ifdefined\figurename
  \renewcommand*\figurename{Figure}
\else
  \newcommand\figurename{Figure}
\fi
\ifdefined\tablename
  \renewcommand*\tablename{Table}
\else
  \newcommand\tablename{Table}
\fi
}
\@ifpackageloaded{float}{}{\usepackage{float}}
\floatstyle{ruled}
\@ifundefined{c@chapter}{\newfloat{codelisting}{h}{lop}}{\newfloat{codelisting}{h}{lop}[chapter]}
\floatname{codelisting}{Listing}
\newcommand*\listoflistings{\listof{codelisting}{List of Listings}}
\makeatother
\makeatletter
\makeatother
\makeatletter
\@ifpackageloaded{caption}{}{\usepackage{caption}}
\@ifpackageloaded{subcaption}{}{\usepackage{subcaption}}
\makeatother
\usepackage{bookmark}
\IfFileExists{xurl.sty}{\usepackage{xurl}}{} % add URL line breaks if available
\urlstyle{same}
\hypersetup{
  pdftitle={Adrian Akhdan Assyauqi},
  pdfauthor={18224055 Adrian Akhdan Assyauqi},
  colorlinks=true,
  linkcolor={blue},
  filecolor={Maroon},
  citecolor={Blue},
  urlcolor={Blue},
  pdfcreator={LaTeX via pandoc}}


\title{Adrian Akhdan Assyauqi}
\usepackage{etoolbox}
\makeatletter
\providecommand{\subtitle}[1]{% add subtitle to \maketitle
  \apptocmd{\@title}{\par {\large #1 \par}}{}{}
}
\makeatother
\subtitle{Portfolio Asesmen II-2100 KIPP}
\author{18224055 Adrian Akhdan Assyauqi}
\date{2025-10-07}
\begin{document}
\maketitle

\renewcommand*\contentsname{Table of contents}
{
\hypersetup{linkcolor=}
\setcounter{tocdepth}{2}
\tableofcontents
}

\bookmarksetup{startatroot}

\chapter*{Selamat Berjumpa}\label{selamat-berjumpa}
\addcontentsline{toc}{chapter}{Selamat Berjumpa}

\markboth{Selamat Berjumpa}{Selamat Berjumpa}

\begin{figure}[H]

{\centering \includegraphics[width=9.5\linewidth,height=\textheight,keepaspectratio]{images/AZRL.png}

}

\caption{About Me}

\end{figure}%

Armein Z R Langi adalah Guru Besar di Sekolah Teknik Elektro dan
Informatika ITB, dosen ITB sejak Desember 1987, mantan Rektor
Universitas Kristen Maranatha, 1 Maret 2016 s/d 29 Februari 2020, mantan
Kepala Pusat Penelitian Teknologi Informasi dan Komunikasi (PP-TIK) ITB
November 2005 s/d Maret 2010, dan Sekretaris MWA ITB Mei 2010-Jan 2011.

Lahir di Tomohon 1962 dari pasangan Manado dan Sunda. Saat ini tinggal
di Bandung, menikah dengan Ina dan dikaruniai empat anak. Ayah dari
Gladys, Kezia, Andria, dan Marco.

Sharing pikiran singkat ada di blog \url{https://azrl.wordpress.com}.
Facebppk: armein\_langi

\bookmarksetup{startatroot}

\chapter{\texorpdfstring{\textbf{Kisah yang Membentuk Diri Anda:
Mengenal Kekuatan Identitas
Naratif}}{Kisah yang Membentuk Diri Anda: Mengenal Kekuatan Identitas Naratif}}\label{kisah-yang-membentuk-diri-anda-mengenal-kekuatan-identitas-naratif}

Manusia adalah pencerita alami. Sejak zaman dahulu, kita selalu berusaha
memahami kekacauan hidup dengan merangkainya menjadi sebuah cerita. Para
ahli bahkan menyebut kita sebagai ``organisme pencerita''
(\emph{storytelling organisms}) yang menjalani ``kehidupan yang penuh
cerita'' (\emph{storied lives}) (2). Proses ini bukanlah sekadar
menyusun fakta, melainkan sebuah proses aktif untuk menciptakan makna.

\href{./audio/Identitas_Naratif__Jadi_Sutradara_dan_Penulis_Kisah_Hidup_Anda_.mp4}{Hayu
dengar poscastnya}

Kisah personal yang terus berkembang inilah yang disebut para psikolog
sebagai \textbf{``identitas naratif''}---sebuah cerita yang kita bangun
untuk memahami keberadaan kita, dengan menghubungkan masa lalu, masa
kini, dan masa depan kita menjadi satu kesatuan yang utuh (3, 8). Cerita
batin ini adalah proses \emph{penciptaan diri} yang aktif, bukan sekadar
menceritakan ulang kejadian. Kisah inilah yang menjawab
pertanyaan-pertanyaan paling mendasar: ``Siapa saya? Bagaimana saya
sampai di sini? Ke mana saya akan pergi?'' (5, 6).

\subsection{\texorpdfstring{\textbf{1. Tiga Lapisan Diri Anda: Di Mana
Cerita Hidup Anda
Berada?}}{1. Tiga Lapisan Diri Anda: Di Mana Cerita Hidup Anda Berada?}}\label{tiga-lapisan-diri-anda-di-mana-cerita-hidup-anda-berada}

Psikolog Dan P. McAdams membagi kepribadian manusia ke dalam tiga
tingkatan yang berbeda. Identitas naratif merupakan tingkatan tertinggi
dan paling personal, yang menyatukan semua bagian lain dari diri kita
(8).

\begin{itemize}
\item
  \textbf{Level 1: Sifat Dasar} Ini adalah ciri-ciri umum kepribadian
  kita yang cenderung stabil, seperti apakah kita seorang
  \emph{introvert} atau \emph{ekstrovert}.
\item
  \textbf{Level 2: Kepedulian Pribadi} Ini mencakup hal-hal yang lebih
  spesifik seperti tujuan hidup, nilai-nilai yang kita pegang, dan
  keyakinan kita.
\item
  \textbf{Level 3: Identitas Naratif} Inilah kisah hidup yang kita
  ciptakan untuk mengikat Level 1 dan 2 menjadi sebuah narasi yang
  koheren dan bermakna. Ini adalah cerita tentang ``diri'' kita.
\end{itemize}

Wawasan paling memberdayakan dari konsep ini adalah: meskipun kita
mungkin tidak dapat dengan mudah mengubah sifat dasar kita (Level 1),
kita \emph{memiliki kekuatan} untuk belajar mengubah cerita yang kita
sampaikan tentang hidup kita (Level 3). Perubahan narasi ini terbukti
memiliki dampak besar pada kesejahteraan dan kebahagiaan kita (9).

Namun, tidak semua cerita diciptakan sama. Mari kita lihat pola-pola
naratif yang dapat membuat sebuah kisah hidup menjadi lebih
memberdayakan.

\subsection{\texorpdfstring{\textbf{2. Pola-Pola Kisah Kehidupan: Apa
yang Membuat Sebuah Cerita
Bermanfaat?}}{2. Pola-Pola Kisah Kehidupan: Apa yang Membuat Sebuah Cerita Bermanfaat?}}\label{pola-pola-kisah-kehidupan-apa-yang-membuat-sebuah-cerita-bermanfaat}

Penelitian menunjukkan bahwa tidak semua cerita yang kita bangun
sama-sama bermanfaat bagi kesehatan mental kita (4). Beberapa tema
naratif secara konsisten terhubung dengan kehidupan yang lebih sejahtera
dan berkembang.

\subsubsection{Penebusan vs.~Kontaminasi: Mengubah Penderitaan Menjadi
Kekuatan}\label{penebusan-vs.-kontaminasi-mengubah-penderitaan-menjadi-kekuatan}

Salah satu pola naratif yang paling penting adalah cara kita membingkai
peristiwa sulit.

\textbf{Cerita Penebusan (Redemption Story)} adalah narasi yang bergerak
dari situasi negatif ke hasil yang positif (misalnya, kegagalan yang
memberikan pelajaran berharga, atau penderitaan yang melahirkan kekuatan
baru). Pola ini sangat kuat kaitannya dengan kebahagiaan, kepuasan
hidup, dan resiliensi (7).

\textbf{Cerita Kontaminasi (Contamination Story)} adalah kebalikannya.
Cerita ini dimulai dari peristiwa baik yang kemudian berubah menjadi
buruk. Kisah semacam ini seperti ``tumpahan minyak yang meracuni air,''
menjebak sang pencerita dalam rasa sakit dan putus asa (3, 11).

\subsubsection{Agensi vs.~Kepasifan: Menjadi Pahlawan dalam Kisah
Anda}\label{agensi-vs.-kepasifan-menjadi-pahlawan-dalam-kisah-anda}

Pola penting lainnya adalah peran yang kita ambil dalam cerita kita
sendiri.

\textbf{Agensi (Agency)} adalah ketika kita menampilkan diri sebagai
aktor utama dalam cerita kita---seseorang yang secara aktif membuat
keputusan, mengambil tindakan, dan mengatasi rintangan. Mengembangkan
rasa agensi dalam cerita hidup adalah salah satu prediktor terkuat untuk
perbaikan dalam terapi (3, 7).

\textbf{Kepasifan (Passivity)} ditandai dengan perasaan menjadi korban
keadaan. Dalam narasi ini, peristiwa seolah-olah ``terjadi begitu saja
pada'' sang pencerita, yang digambarkan sebagai korban pasif dari takdir
atau tindakan orang lain (8).

Tabel berikut merangkum tema-tema naratif yang membangun dan merusak,
beserta dampaknya bagi kesejahteraan kita.

\begin{longtable}[]{@{}
  >{\raggedright\arraybackslash}p{(\linewidth - 2\tabcolsep) * \real{0.5000}}
  >{\raggedright\arraybackslash}p{(\linewidth - 2\tabcolsep) * \real{0.5000}}@{}}
\toprule\noalign{}
\endhead
\bottomrule\noalign{}
\endlastfoot
Pola Naratif & Dampak Psikologis \\
\textbf{Pola Naratif yang Membangun (Generative Themes)} & \\
\textbf{Penebusan} (Negatif → Positif) & Meningkatkan kebahagiaan,
kepuasan hidup, resiliensi, dan \emph{generativitas} (keinginan untuk
berkontribusi pada kesejahteraan generasi mendatang) (4, 7). \\
\textbf{Agensi} (Diri sebagai Aktor Efektif) & Meningkatkan kepercayaan
diri, kesehatan mental, dan merupakan prediktor kuat perbaikan dalam
terapi (7). \\
\textbf{Koneksi} (Hubungan \& Rasa Memiliki) & Meningkatkan
kesejahteraan, mengurangi rasa kesepian, dan memberikan rasa memiliki
tujuan hidup yang lebih besar (8). \\
\textbf{Pola Naratif yang Merusak (Disruptive Themes)} & \\
\textbf{Kontaminasi} (Positif → Negatif) & Menurunkan kesejahteraan,
menyebabkan depresi, keputusasaan, dan perasaan terperangkap dalam
pengalaman negatif (3). \\
\textbf{Kepasifan} (Diri sebagai Korban) & Menimbulkan perasaan menjadi
korban, demotivasi, rasa tidak berdaya, depresi, dan hasil kesehatan
mental yang buruk (8). \\
\textbf{Isolasi} (Terputus dari Orang Lain) & Menyebabkan kesepian,
keputusasaan, kurangnya dukungan sosial, dan meningkatkan kerentanan
terhadap gangguan psikologis (8). \\
\end{longtable}

Lalu, bagaimana pikiran kita menciptakan pola-pola naratif ini?
Jawabannya terletak pada sebuah proses kognitif yang luar biasa.

\subsection{\texorpdfstring{\textbf{3. Seni Memberi Makna: Kekuatan
Super Anda dalam
Bernalar}}{3. Seni Memberi Makna: Kekuatan Super Anda dalam Bernalar}}\label{seni-memberi-makna-kekuatan-super-anda-dalam-bernalar}

Pikiran kita memiliki ``mesin pembuat makna'' yang disebut
\textbf{penalaran otobiografis} (\emph{autobiographical reasoning}).
Inilah kemampuan kognitif yang memungkinkan kita menghubungkan
peristiwa-peristiwa dalam hidup dengan identitas diri kita dan memahami
signifikansinya (10).

Tanpa penalaran ini, hidup kita hanyalah daftar kejadian. Dengan
penalaran ini, hidup kita menjadi sebuah cerita yang bermakna.

Temuan paling penting dari psikologi naratif adalah ini:
\textbf{kemampuan kita untuk memaknai peristiwa sulit secara positif
(misalnya, menemukan hikmah atau pelajaran) lebih berpengaruh pada
kesejahteraan kita daripada peristiwa itu sendiri} (10). Ini bukan sifat
bawaan, melainkan sebuah keterampilan yang bisa dipelajari dan dilatih.

Memahami hal ini memberi kita kekuatan. Langkah-langkah berikutnya
adalah latihan praktis untuk mengasah `mesin pembuat makna' ini dan
menjadi penulis yang lebih sadar atas kisah hidup kita sendiri.

\subsection{\texorpdfstring{\textbf{4. Mulai Menulis Ulang Kisah Anda:
Dua Langkah
Praktis}}{4. Mulai Menulis Ulang Kisah Anda: Dua Langkah Praktis}}\label{mulai-menulis-ulang-kisah-anda-dua-langkah-praktis}

Meskipun kita tidak bisa mengubah masa lalu, kita memiliki kekuatan luar
biasa untuk mengubah \emph{cerita} yang kita sampaikan tentang masa lalu
itu (29). Berikut adalah dua langkah praktis yang terinspirasi dari
Terapi Naratif untuk memulai proses ini.

\subsubsection{Langkah 1: Pisahkan Diri Anda dari
Masalah}\label{langkah-1-pisahkan-diri-anda-dari-masalah}

Teknik ini disebut \textbf{eksternalisasi masalah}. Caranya adalah
dengan mengubah cara kita berbicara tentang masalah kita.

Misalnya, alih-alih berpikir, \emph{``Saya adalah orang yang
pencemas,''} coba bingkai ulang menjadi, \emph{``Saya adalah orang yang
sedang berhadapan dengan pengaruh kecemasan''} (35).

Pergeseran bahasa yang sederhana ini menciptakan jarak psikologis.
Masalah tidak lagi menjadi bagian inti dari identitas Anda, melainkan
sesuatu di luar diri Anda yang bisa diamati, dipahami, dan dihadapi. Ini
membuat masalah terasa jauh lebih bisa dikelola.

\subsubsection{Langkah 2: Temukan ``Momen Berkilau''
Anda}\label{langkah-2-temukan-momen-berkilau-anda}

Setelah Anda memisahkan diri dari masalah, langkah selanjutnya adalah
mencari bukti yang bertentangan dengan ``cerita yang penuh masalah''
tersebut. Dalam Terapi Naratif, ini disebut \emph{unique outcomes} atau
yang bisa kita sebut \textbf{momen berkilau} (\emph{sparkling moments}).

Ini adalah momen-momen, sekecil apa pun, di mana masalah tersebut tidak
berkuasa atas diri Anda. Tanyakan pada diri Anda:

``Ingatkah saat di mana `Si Pengkritik' dalam diri Anda muncul, tetapi
Anda tetap berhasil bertindak dengan percaya diri?'' (36)

Atau, ``Apakah ada momen ketika `rasa malas' mencoba mengambil alih,
tetapi Anda tetap berhasil menyelesaikan tugas itu?''

Momen-momen berkilau ini adalah bukti nyata dari kekuatan, nilai, dan
ketahanan Anda. Mereka adalah bahan mentah yang dapat Anda gunakan untuk
mulai menenun sebuah cerita baru yang lebih kuat dan lebih
memberdayakan.

\subsection{\texorpdfstring{\textbf{5. Kesimpulan: Kisah Anda Adalah
Perjalanan yang Terus
Berlanjut}}{5. Kesimpulan: Kisah Anda Adalah Perjalanan yang Terus Berlanjut}}\label{kesimpulan-kisah-anda-adalah-perjalanan-yang-terus-berlanjut}

Pada akhirnya, kita semua adalah penulis kisah hidup kita sendiri.
Cerita yang kita sampaikan kepada diri kita sendiri secara aktif
menciptakan realitas kita (5). Kisah hidup yang sehat ditandai oleh
tema-tema penebusan, di mana kesulitan diubah menjadi pertumbuhan, dan
agensi, di mana kita menjadi pahlawan dalam perjalanan kita sendiri.

Tujuannya bukanlah untuk menulis sebuah cerita yang ``sempurna'' dan
tanpa cela. Tujuannya adalah untuk menumbuhkan keberanian untuk terus
menulis, terus mencari makna, dan menjadi penulis sebuah kisah hidup
yang berani, jujur, dan layak untuk diceritakan.

\bookmarksetup{startatroot}

\chapter{My Songs for You}\label{my-songs-for-you}

Post Wedding Kawah Putih Lirik by Armein Z. R. Langi Music: SUNO

\url{https://youtu.be/KWthEImJ9mY?si=iUV8Ghhj0R3gKQyZ}

Why am I singing for you? \href{./Rivers\%20In\%20My\%20Mind.mp3}{River
in my Mind}

Falling in love everyday \href{./Heaven\%20on\%20Earth.mp3}{Heaven on
Earth}

\bookmarksetup{startatroot}

\chapter{My Stories for You}\label{my-stories-for-you}

A story about my oldest daughter
\href{https://azrl.wordpress.com/2020/07/18/gaun-pengantin-gladys/\#comment-28004}{Gaun
Pemngantin Gladys}

A message to my daughter
\href{https://azrl.wordpress.com/2021/10/06/the-child-who-learned-to-walk-at-the-disneyland/}{The
Child Who Learned to Walk at the Disneyland}

A story for my students
\href{https://azrl.wordpress.com/2008/04/21/fly-my-eagle-fly/}{Fly Eagle
Fly}

A (true) story for my teachers
\href{\%3Chttps://azrl.wordpress.com/2012/11/28/perginya-sang-mahaputera-dan-mahaguru-berkemeja-putih/}{Sang
Mahaguru, Sang Mahaputera}

Teasing story \url{https://www.youtube.com/watch?v=Dg_4PbBlBf4}

\bookmarksetup{startatroot}

\chapter{My Shapes}\label{my-shapes}

\bookmarksetup{startatroot}

\chapter{My Personal Reviews}\label{my-personal-reviews}

Berikut cara saya melakukan review
\href{./Doc.5.Mengevaluasi-Esai-Berdasarkan-Rubrik.pdf}{Menilai dan
Mengevaluasi Esai Berdasarkan Rubrik}

\bookmarksetup{startatroot}

\chapter{My Concepts}\label{my-concepts}

Mau hidup epik ? \href{lifestory.pdf}{Write your Life Story}

Apa itu berkonsep?

\url{https://youtu.be/QVfUlVBO80U?si=yM6q_rwV9rcDBbu7}

\bookmarksetup{startatroot}

\chapter{My Opinions}\label{my-opinions}

SApa itu beropini? \href{BM\%20Opini.mp4}{Opini Berpengaruh}

Bagiamana menjaadi menarik? \href{./Interesting.mp4}{Menjadi Menarik}

\bookmarksetup{startatroot}

\chapter{My Innovations}\label{my-innovations}

\bookmarksetup{startatroot}

\chapter{My Knowledge}\label{my-knowledge}

Cara saya mengkomunikasikan sebuah pengatahuan sebagai petunjuk bagi
orang lain 1) saya tulis
\href{Rekomendasi\%20Presentasi\%20Efektif(Contoh\%20Makalah).pdf}{makalah
sebagai bahan utama} 2) lalu saya buat
\href{Contoh\%20Transkrip\%20Presentasi.pdf}{transkrip ucapan lisan} 3)
kemudian saya kembangkan
\href{Rekomendasi\%20Presentasi\%20(Contoh\%20Slides).pdf}{slide
pendukung trnsskrip} 4) lalu saya memproduksivideo audio visual
\url{https://youtu.be/ZbghfMvnPZc} \url{https://youtu.be/ZbghfMvnPZc}

\bookmarksetup{startatroot}

\chapter{My Professional Reviews}\label{my-professional-reviews}

Untuk melkukan review, seperti pada
\href{../My_Personal_Reviews/Doc.5.Mengevaluasi-Esai-Berdasarkan-Rubrik.pdf}{pendekatan
AI}, kita membutuhkan rubrik - Rubrik
\href{Dok.4.a.Rubrik_Kisah.pdf}{Kisah} - Rubrik
\href{Dok.4.b.Rubrik_Konsep.pdf}{Konsep} - Rubrik
\href{Dok.4.c.Rubrik_Opini.pdf}{Opini}

\bookmarksetup{startatroot}

\chapter{Summary}\label{summary}

In summary, this book has no content whatsoever.

\bookmarksetup{startatroot}

\chapter*{References}\label{references}
\addcontentsline{toc}{chapter}{References}

\markboth{References}{References}

\phantomsection\label{refs}




\end{document}
